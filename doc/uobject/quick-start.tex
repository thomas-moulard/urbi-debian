%% Copyright (C) 2009-2010, Gostai S.A.S.
%%
%% This software is provided "as is" without warranty of any kind,
%% either expressed or implied, including but not limited to the
%% implied warranties of fitness for a particular purpose.
%%
%% See the LICENSE file for more information.

\chapter{Quick Start}
\label{sec:uob:quick}

This chapter presents \usdk with a specific focus on its
middleware features.  It is self-contained in order to help readers
quickly grasp the potential of \urbi used as a middleware.  References
to other sections of this document are liberally provided to point the
reader to the more complete documentation; they should be ignored
during the first reading.

\section{UObject Basics}

As a simple running example, consider a (very) basic factory.  Raw
material delivered to the factory is pushed into some assembly machine,
which takes some time.

As a firth component of this factory, the core engine of the factory
is implemented as follows.  This class is pure regular \Cxx, it uses
no \urbi feature at all.

\newcommand{\factoryDir}{\uobjectsDir/factory.uob}
\lstinputlisting[language=C++,basicstyle=\ttfamily\footnotesize]{\factoryDir/factory.hh}
\lstinputlisting[language=C++,basicstyle=\ttfamily\footnotesize]{\factoryDir/factory.cc}

\subsection{Wrapping into an UObject}

By \dfn{binding} a UObject, we mean using the UObject API to declare
objects to the \urbi world.  These objects have member variables (also
known as \dfn{attributes}) and/or member functions (also known as
\dfn{methods}) all of them or some of them being declared into the
\urbi world.

One could modify the \lstinline|Factory| class to make it a UObject,
yet we rather recommend wrapping pure \Cxx classes into a UObject.  It
is strongly suggested to aggregate the native \Cxx objects in the
UObject --- rather than trying to derive from it.  By convention, we
prepend a ``U'' to the name of the base class, hence the
\lstinline|UFactory| class.

\lstinputlisting[language=C++,basicstyle=\ttfamily\footnotesize]{\factoryDir/ufactory.hh}

The implementation of \lstinline|UFactory| is quite straightforward.
It uses some of the primitives used in the binding process
(\autoref{sec:uob:api:bind}):
\begin{itemize}
\item \lstinline|UStart(\var{class})| declares classes that are
  UObjects; eventually such classes will appear in \us as
  \lstinline|uobjects.\var{class}|.  Use it once.

\item \lstinline|UBindFunction(\var{class}, \var{function})| declares
  a \var{function}.  Eventually bound in the \us world as
  \lstinline|uobjects.\var{class}.\var{function}|.

\item Similarly, \lstinline|UBindVar(\var{class}, \var{variable})|
  declares a \var{variable}.
\end{itemize}

\urbi relies on the prototype model for object-oriented programming,
which is somewhat different from the traditional \Cxx class-based
model (\autoref{sec:tut:value}).  This is reflected by the presence of
\emph{two} different constructors:
\begin{itemize}
\item \lstinline|UFactory::UFactory|, the \Cxx constructor which is
  invoked for every single instance of the UObject.  It is always
  invoked by the \urbi system when instantiating a UObject,
  \emph{including} the prototype itself.  It sole argument is its name
  (an internal detail you need not be aware of), and its main purpose
  is to register attributes and functions, especially the
  \lstinline|UFactory::init| function.

\item \lstinline|UFactory::init|, the \urbi constructor invoked each
  time a new clone of \lstinline|UFactory| is made, i.e., for every
  instance except the first one.

  Functions and variables that do not make sense for the initial
  prototype (which might not be fully functional) should be bound
  here, rather that in the \Cxx constructor.
\end{itemize}

The following listing is abundantly commented, and is easy to grasp.

\lstinputlisting[language=C++,basicstyle=\ttfamily\footnotesize]{\factoryDir/ufactory.cc}

\subsection{Running Components}
\label{sec:uob:quick:run}
%  * Compilation et branchement de UObject (urbi-launch): expliquer les
%  deux modes: distant / pluge'
%
%  * Cycle complet de lancement d'une appli Urbi: lancement d'urbi,
%  execution d'urbi.ini, chargement dynamique des UObjects


As a first benefit from using the \urbi environment, this source code
is already runnable!  No \lstinline|main| function is needed, the
\urbi system provides one.

\subsubsection{Compiling}
Of course, beforehand, we need to compile this UObject into some loadable
module.  The \urbi modules are \dfn{shared objects}, i.e., libraries that
can be loaded on demand (and unloaded) during the execution of the program.
Their typical file names depend on the architecture: \file{ufactory.so} on
most Unix (including Mac OS X), and \file{ufactory.dll} on Windows.  To
abstract away from these differences, we will simply use the base name,
\file{ufactory} with the \urbi tool chain.

There are several options to compile our factory as a UObject.  If you are
using Microsoft Visual Studio, the \usdk installer created a ``UObject''
project template accessible through the ``New project'' wizard.  Otherwise,
you can use directly your regular compiler tool chain.  You may also use
\command{umake-shared} from the \file{umake-*} family of programs
(\autoref{sec:tools:umake:wrappers}):

\begin{shell}
$ ls factory.uob
factory.cc  factory.hh  ufactory.cc  ufactory.hh
$ umake-shared factory.uob -o factory
# ... Lots of compilation log messages ...
$ ls
_ubuild-factory.so  factory.la  factory.so  factory.uob
\end{shell}

The various files are:
\begin{description}
\item[\file{factory.uob}] Merely by convention, the sources of our
  UObject are in a \file{*.uob} directory.
\item[\file{factory.so}] A shared dlopen-module.  This is the ``true''
  product of the \command{umake-shared} invocation.  Its default name
  can be quite complex (\file{uobject-i386-apple-darwin9.7.0.so} on my
  machine), as it will encode information about the architecture of
  your machine; if you don't need such accuracy, use the option
  \option{--output}/\option{-o} to specify the output file name.

  \command{umake-shared} traversed \file{factory.uob} to gather and
  process relevant files (source files, headers, libraries and object
  files) in order to produce this output file.

\item[\file{\_ubuild-factory.so}] this is a temporary directory in
  which the compilation takes place.  It can be safely removed by
  hand, or using \command{umake-deepclean}
  (\autoref{sec:tools:umake:wrappers}).

\item[\file{factory.la}] a
  \href{http://www.gnu.org/software/libtool/}{GNU Libtool} file that
  contains information such as dependencies on other libraries.  While
  this file should be useless most of the time, we recommend against
  removing it: it may help understand some problems.
\end{description}

\subsubsection{Running UObjects}

There are several means to toy with this simple UObject.  You can use
\command{urbi-launch} (\autoref{sec:tools:urbi-launch}) to plug the
UFactory in an \urbi server and enter an interactive session.

\begin{shell}[alsolanguage={[interactive]Urbi}]
# Launch an Urbi server with UFactory plugged in.
$ urbi-launch --start ufactory -- --interactive
[00000987] *** ********************************************************
[00000990] *** Urbi SDK version 2.0.3 rev. d6a568d
[00001111] *** Copyright (C) 2005-2010 Gostai S.A.S.
[00001111] ***
[00001111] *** This program comes with ABSOLUTELY NO WARRANTY.  It can
[00001112] *** be used under certain conditions.  Type `license;',
[00001112] *** `authors;', or `copyright;' for more information.
[00001112] ***
[00001112] *** Check our community site: http://www.urbiforge.org.
[00001112] *** ********************************************************
var f = uobjects.UFactory.new(1s);
[00020853] UFactory_0x1899c90
f.assemble(["Hello, ", "World!"]);
[00038705] "Hello, World!"
shutdown;
\end{shell}

You may also launch the factory UObject in the background, as a
network component:

\begin{shell}[alsolanguage={[interactive]Urbi}]
$ urbi-launch --start ufactory --host 0.0.0.0 --port 54000 &
\end{shell}

\noindent
and interact with it using your favorite client (\command{telnet},
\command{netcat}, \command{socat}, \ldots), or using the
\command{urbi-send} tool (\autoref{sec:tools:urbi-send}).

\begin{shell}[alsolanguage={[interactive]Urbi}]
$ urbi-send --port 54000                            \
            -e 'var f = uobjects.UFactory.new(1s)|' \
            -e 'f.assemble(["12", "34"]);'          \
            --quit
[00146148] "1234"
[00000000:client_error] End of file
$ urbi-send --port 54000                                            \
            -e uobjects.UFactory.new(1s).assemble(["34", "12"]);'   \
            --quit
[00146148] "3412"
[00000000:client_error] End of file
\end{shell}


\subsection{Asynchronicity}
% * Asynchronisme dans UObject: notifychange, timers


\section{Using \us}

% Introduction a` Urbiscript (rappeler que c'est avant tout un langage
% comme les autres, avec if/for/while, etc), puis passer rapidement
% aux elements clefs de la prog evenementielle dans Urbi: at/whenever

\us is a programming language primarily designed for robotics.  Its
syntax is inspired by that of \Cxx: if you know \C, \Cxx, \java or \Cs,
writing \us programs is easy.  It's a dynamic object-oriented
(\autoref{sec:tut:object}) scripting language, which makes it well
suited for high-level application.  It supports and emphasizes
parallel (\autoref{sec:tut:concurrent}) and event-based programming
(\autoref{sec:tut:event-prog}), which are very popular paradigms in
robotics, by providing core primitives and language constructs.

Thanks to its client/server approach, one can easily interact with a
robot, to monitor it, or to experiment live changes in the \us
programs.

Courtesy of the UObject architecture, \us is fully integrated with
\Cxx. As already seen in the above examples
(\autoref{sec:uob:quick:run}), you can bind \Cxx classes in \us
seamlessly. \us is also integrated with many other languages such as
\java, \matlab or \python.  UObject also naturally provide \us with
support for distributed architectures: objects can run in different
processes, possibly on remote computers.

%% Everything that follows is bullshit: UObject invocations are
%% *not* in different threads, so we are definitely doomed here.
%%
\subsection{The \us Scripting Language}

The following example shows how one can easily interface UObjects into
the \us language.  The following simple class (actually, a genuine
object, in \us ``classes are objects'', see \autoref{sec:tut:object})
aggregates two assembly machines, a fast one, and a slow one.  This
class demonstrates usual object-oriented, sequential, features.

\begin{urbiscript}[firstnumber=1]
//#plug urbi/factory
class TwoMachineFactory
{
  // A shorthand common to all the Two Machine factories.
  var UFactory = uobjects.UFactory;

  // Default machines.
  var this.fastMachine = UFactory.new(10ms);
  var this.slowMachine = UFactory.new(100ms);

  // The urbiscript constructor.
  // Build two machines, a fast one, and a slow one.
  function init(fast = 10ms, slow = 100ms)
  {
    // Make sure fast <= slow.
    if (slow < fast)
      [fast, slow] = [slow, fast];
    // Two machines for each instance of factory.
    fastMachine = UFactory.new(fast);
    slowMachine = UFactory.new(slow);
  };

  // Wrappers to make invocation of the machine simpler.
  function fast(input) { fastMachine.assemble(input) };
  function slow(input) { slowMachine.assemble(input) };

  // Use the slow machine for large jobs.
  function assemble(input)
  {
    var res;
    if (5 < input.size)
      { res = slow(input); echo("Used the slow machine") }
    else
      { res = fast(input); echo("Used the fast machine") } |
    res
  };
};
[00000001] TwoMachineFactory
\end{urbiscript}

Using this class is straightforward.

\begin{urbiscript}
var f = TwoMachineFactory.new|;
f.assemble([1, 2, 3, 4, 5, 6]);
[00000002] *** Used the slow machine
[00000003] "123456"
f.assemble([1]);
[00000004] *** Used the fast machine
[00000005] "1"
\end{urbiscript}

The genuine power of \us is when concurrency comes into play.

\subsection{Concurrency}
% * Parallelisme explicite: \&
%
% * Tags et controle d'execution
%
% * Channels et liburbi (les principes, renvoyer a` la doc liburbi pour
% l'API)
%
% * All together, on reprend l'exemple du de'but et on de'taille le code
% des UObjects utilise's, puis le code urbiscript.

First of all, why should we wait for the slow job to finish if we have
a fast machine available?

\begin{urbiscript}
f.assemble([1, 2, 3, 4, 5, 6]),
f.assemble([1]),
[00000004] *** Used the fast machine
sleep(100ms);
[00000002] *** Used the slow machine
\end{urbiscript}

Contrary to the previous run, the fast machine answered first.  There
is one bit (one dot?) of difference between both runs: the first one
used the sequential terminator, \samp{;}, while the second run used
\samp{,} to send the execution ``in background''.

You maybe also notice that the result is no longer reported.  Indeed,
the \us interactive shell only displays the result of synchronous
expressions (i.e., those ending with a \samp{;}): asynchronous answers
are confusing (see the inversion here).

Channels (\refObject{Channel}) are useful to ``send'' asynchronous
answers.

\begin{urbiassert}
//for, (var list = [1, 2, 3, 4, 5, 6, 7];
//      !list.empty;
//      list.removeBack)
//  cout << list.asString + " => " + f.assemble(list);
//sleep(1s);
\end{urbiassert}

\section{Conclusion}
% * Conclusion note: expliquer qu'on peut embedder l'engine Urbi dans
% une appli native C++ (tout le monde le demande), qu'on peut faire
% des bridges ge'ne'riques avec d'autres archis a` composants (citer
% CORBA), donner le lien pour te'le'charger, le lien vers la doc du
% langage urbiscript, vers la doc de UObject, etc.



%%% Local Variables:
%%% mode: latex
%%% TeX-master: "../urbi-sdk"
%%% ispell-dictionary: "american"
%%% ispell-personal-dictionary: "../urbi.dict"
%%% fill-column: 76
%%% End:
