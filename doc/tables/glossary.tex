%% Copyright (C) 2010, Gostai S.A.S.
%%
%% This software is provided "as is" without warranty of any kind,
%% either expressed or implied, including but not limited to the
%% implied warranties of fitness for a particular purpose.
%%
%% See the LICENSE file for more information.

\chapter{Glossary}
\label{sec:glossary}

This chapter aggregates the definitions used in the this document.

% \screenshot{FILE-and-LABEL}{CAPTION}
% ------------------------------------
\newcommand{\screenshot}[2]
{
  \begin{figure}[htp]
    \centering
    \includegraphics[width=.8\linewidth]{img/#1}
    \caption{#2}
    \label{fig:#1}
  \end{figure}
}

\begin{description}
\item[Bioloid] The Robotis Bioloid is a hobbyist and educational robot kit
  produced by the Korean robot manufacturer Robotis. The Bioloid platform
  consists of components and small, modular servomechanisms called
  Dynamixels, which can be used in a daisy-chained fashion to construct
  robots of various configurations, such as wheeled, legged, or humanoid
  robots. The Bioloid system is thus comparable to the LEGO Mindstorms and
  VEXplorer kits.

\item[Gostai Console] This tool provides a graphical user interface to a
  remote \urbi server (see \autoref{fig:gostai-console}).  Unix users
  (GNU/Linux or Mac OS X) can use the traditional \command{telnet} tool.
  Windows users are invited to use Gostai Console instead.  See
  \autoref{sec:tut:started}.

  \screenshot{gostai-console}{Gostai Console}

\item[Gostai Editor] Also called UEdit, Gostai \us Editor
  (\autoref{fig:gostai-editor}) is a lightweight \us source code editor with
  semantical highlighting. It is available for Windows and GNU/Linux
  platforms.

  \screenshot{gostai-editor}{Gostai Editor}

\item[Gostai Lab] This tool (see \autoref{fig:gostai-lab}), which
  includes the features of Gostai Console, allows to build easily
  elaborate remote controller for robots.  It provides various widgets
  to visualize data from the robot (including video and sound), and to
  modify the state of the robot.

  \screenshot{gostai-lab}{Gostai Lab}

\item[Gostai Studio] This tool (see \autoref{fig:gostai-studio}),
  includes all the features of Gostai Console and Gostai Lab.  It is a
  high-level Integrated Development Environment for \urbi.  Its
  formalism is based on \dfn{Hierarchical Finite State Machines}.

  \screenshot{gostai-studio}{Gostai Studio}

\item[RMP] The Segway Robotic Mobility Platform is a robotic platform based
  on the Segway Personal Transporter.  See \url{http://rmp.segway.com}.

\item[ROS] Robot Operating System, \url{http://www.ros.org/}, developed by
  Willow Garage, \url{http://www.willowgarage.com/}.  It is an abstraction
  layer on top of the genuine operating system (such as GNU/Linux) that
  provides hardware abstraction, device control, common algorithms,
  message-passing between processes, and package management.

\item[Spykee] The Spykee is a WiFi-enabled robot built by Meccano (known as
  Erector in the United States). It is equipped with a camera, speaker,
  microphone, and moves using two tracks. See
  \url{http://www.spykeeworld.com}.

\item[urbi-console] Former name of ``Gostai Console''.  See that item.
\end{description}

%%% Local Variables:
%%% mode: latex
%%% TeX-master: "urbi-sdk"
%%% ispell-dictionary: "american"
%%% ispell-personal-dictionary: "../urbi.dict"
%%% fill-column: 76
%%% End:
