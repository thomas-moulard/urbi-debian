%% Copyright (C) 2009-2010, Gostai S.A.S.
%%
%% This software is provided "as is" without warranty of any kind,
%% either expressed or implied, including but not limited to the
%% implied warranties of fitness for a particular purpose.
%%
%% See the LICENSE file for more information.

\chapter{Introduction}

\usdk is a fully-featured environment to orchestrate complex
organizations of components.  It relies on a middleware architecture
that coordinates components named UObjects.  It also features \us, a
scripting language that can be used to write orchestration programs.

\section{\urbi and UObjects}

\urbi makes the orchestration of independent, concurrent, components
easier.  It was first designed for robotics: it provides all the
needed features to coordinate the execution of various components
(actuators, sensors, software devices that provide features such as
text-to-speech, face recognition and so forth).  Languages such as
\Cxx are well suited to program the local, low-level, handling of
these hardware or software devices; indeed one needs efficiency, small
memory footprint, and access to low-level hardware details.  Yet, when
it comes to component orchestration and coordination, in a word, when
it comes to addressing concurrency, it can be tedious to use such
languages.

Middleware infrastructures make possible to use remote components as
if they were local, to allow concurrent execution, to make synchronous
or asynchronous requests and so forth.  The \dfn{UObject} \Cxx
architecture provides exactly this: a common API that allows
conforming components to be used seamlessly in highly concurrent
settings.  Components need not be designed with UObjects in mind,
rather, UObjects are typically ``shells'' around ``regular''
components.

Components with an UObject interface are naturally supported by the
\us programming language.  This provides a tremendous help: one can
interact with these components (making queries, changing them,
observing their state, monitoring various kinds of events and so
forth), which provides a huge speed-up during development.

Finally, note that, although made with robots in mind, the UObject
architecture is well suited to tame any heavily concurrent
environment, such as video games or complex systems in general.

\section{\urbi and \us}

\us is a programming language primarily designed for robotics. It's a
dynamic, prototype-based, object-oriented scripting language. It
supports and emphasizes parallel and event-based programming, which
are very popular paradigms in robotics, by providing core primitives
and language constructs.

Its main features are:
\begin{itemize}
\item syntactically close to \Cxx. If you know \C or \Cxx, you can
  easily write \us programs.
\item fully integrated with \Cxx. You can bind \Cxx classes in \us
  seamlessly. \us is also integrated with many other languages such as
  \java, \matlab or \python.
\item object-oriented. It supports encapsulation, inheritance and
  inclusion polymorphism. Dynamic dispatching is available through
  monomethods --- just as \Cxx, \Cs or \java.
\item concurrent. It provides you with natural constructs to run and
  control high numbers of interacting concurrent tasks.
\item event-based. Triggering events and reacting to them is
  absolutely straightforward.
\item functional programming.  Inspired by languages such as \lisp or
  \caml, \us features first class functions and pattern matching.
\item client/server.  The interpreter accepts multiple connections
  from different sources (human users, robots, other servers \ldots)
  and enables them to interact.
\item distributed.  You can run objects in different processes,
  potentially remote computers across the network.
\end{itemize}

\section{Genesis}

\urbi what first designed and implemented by Jean-Christophe Baillie,
together with Matthieu Nottale.  Because its users wildly acclaimed it,
Jean-Christophe founded Gostai, a France-based Company that develops
software for robotics with a strong emphasis on personal robotics.

\paragraph{Authors}
\usdk 1 was further developed by Akim Demaille, Guillaume Deslandes, Quentin
Hocquet, and Benoît Sigoure.

The \usdk 2 project was started and developed by Akim Demaille, Quentin
Hocquet, Matthieu Nottale, and Benoît Sigoure.  Samuel Tardieu provided an
immense help during the year 2008, in particular for the concurrency and
event support.

The maintenance is currently carried out by Akim Demaille, Quentin
Hocquet, and Matthieu Nottale.  Jean-Christophe Baillie is still
deeply involved in the development of \us, he regularly submits ideas,
and occasionally even code!

\paragraph{Contributors}

A number of people contributed significantly to \urbi, including Romain
Bezut, Thomas Moulard, Nicolas Pierron.

\section{Outline}

This multi-part document provides a complete guide to Urbi.  See
\autoref{sec:notations} for the various notations that are used in the
document.

\newenvironment{partDescription}[2]
{%
  \item[\autoref{#1} --- \nameref{#1}]~\\%
  #2
  \begin{description}%
    \let\itemOrig\item%
    \renewcommand{\item}[1][]{\itemOrig[~~\autoref{##1} --- \nameref{##1}]~\\}%
  }{%
  \end{description}%
}

%%% Keep sync with urbi-sdk.tex.
\begin{description}
%% Copyright (C) 2010, Gostai S.A.S.
%%
%% This software is provided "as is" without warranty of any kind,
%% either expressed or implied, including but not limited to the
%% implied warranties of fitness for a particular purpose.
%%
%% See the LICENSE file for more information.

\begin{partDescription}{part:specs}
  {
    %
    This part defines the specifications of the \us language version
    2.0. It defines the expected behavior from the \us interpreter,
    the standard library, and the \sdk. It can be used to check
    whether some code is valid, or browse \us or \Cxx \api for a
    desired feature. Random reading can also provide you with advanced
    knowledge or subtleties about some \us aspects.

    This part is not an \us tutorial; it is not structured in a
    progressive manner and is too detailed.  Think of it as a
    dictionary: one does not learn a foreign language by reading a
    dictionary. The \us Tutorial (\autoref{part:tut}), or the live \us
    tutorial built in the interpreter are good introductions to \us.

    This part does not aim at giving advanced programming
    techniques. Its only goal is to define the language and its
    libraries.
    %
  }
\item[sec:tools]
  Presentation and usage of the different tools available with the
  \urbi framework related to \us, such as the \urbi server, the
  command line client, \umake, \ldots

\item[sec:lang]
  Core constructs of the language and their behavior.

\item[sec:stdlib]
  Listing of all classes and methods provided in the standard library.

\item[sec:specs:ros] \urbi provides a set of tools to communicate with ROS
  (Robot Operating System). For more information about ROS, see
  \url{http://www.ros.org}.  \urbi, acting as a ROS node, is able to
  interact with the ROS world.

\item[sec:naming]
  Also known as ``The \urbi Naming Standard'': naming conventions in
  for standard hardware/software devices and components implemented as
  UObject and the corresponding slots/events to access them.

%\item[sec:sdk]
%  The \urbi software development kit that enable to
%  interact with \urbi from \Cxx.
\end{partDescription}


%%% Local Variables:
%%% mode: latex
%%% TeX-master: "../urbi-sdk"
%%% ispell-dictionary: "american"
%%% ispell-personal-dictionary: "../urbi.dict"
%%% fill-column: 76
%%% End:

%% Copyright (C) 2010, Gostai S.A.S.
%%
%% This software is provided "as is" without warranty of any kind,
%% either expressed or implied, including but not limited to the
%% implied warranties of fitness for a particular purpose.
%%
%% See the LICENSE file for more information.

\begin{partDescription}{part:specs}
  {
    %
    This part defines the specifications of the \us language version
    2.0. It defines the expected behavior from the \us interpreter,
    the standard library, and the \sdk. It can be used to check
    whether some code is valid, or browse \us or \Cxx \api for a
    desired feature. Random reading can also provide you with advanced
    knowledge or subtleties about some \us aspects.

    This part is not an \us tutorial; it is not structured in a
    progressive manner and is too detailed.  Think of it as a
    dictionary: one does not learn a foreign language by reading a
    dictionary. The \us Tutorial (\autoref{part:tut}), or the live \us
    tutorial built in the interpreter are good introductions to \us.

    This part does not aim at giving advanced programming
    techniques. Its only goal is to define the language and its
    libraries.
    %
  }
\item[sec:tools]
  Presentation and usage of the different tools available with the
  \urbi framework related to \us, such as the \urbi server, the
  command line client, \umake, \ldots

\item[sec:lang]
  Core constructs of the language and their behavior.

\item[sec:stdlib]
  Listing of all classes and methods provided in the standard library.

\item[sec:specs:ros] \urbi provides a set of tools to communicate with ROS
  (Robot Operating System). For more information about ROS, see
  \url{http://www.ros.org}.  \urbi, acting as a ROS node, is able to
  interact with the ROS world.

\item[sec:naming]
  Also known as ``The \urbi Naming Standard'': naming conventions in
  for standard hardware/software devices and components implemented as
  UObject and the corresponding slots/events to access them.

%\item[sec:sdk]
%  The \urbi software development kit that enable to
%  interact with \urbi from \Cxx.
\end{partDescription}


%%% Local Variables:
%%% mode: latex
%%% TeX-master: "../urbi-sdk"
%%% ispell-dictionary: "american"
%%% ispell-personal-dictionary: "../urbi.dict"
%%% fill-column: 76
%%% End:

%% Copyright (C) 2010, Gostai S.A.S.
%%
%% This software is provided "as is" without warranty of any kind,
%% either expressed or implied, including but not limited to the
%% implied warranties of fitness for a particular purpose.
%%
%% See the LICENSE file for more information.

\begin{partDescription}{part:specs}
  {
    %
    This part defines the specifications of the \us language version
    2.0. It defines the expected behavior from the \us interpreter,
    the standard library, and the \sdk. It can be used to check
    whether some code is valid, or browse \us or \Cxx \api for a
    desired feature. Random reading can also provide you with advanced
    knowledge or subtleties about some \us aspects.

    This part is not an \us tutorial; it is not structured in a
    progressive manner and is too detailed.  Think of it as a
    dictionary: one does not learn a foreign language by reading a
    dictionary. The \us Tutorial (\autoref{part:tut}), or the live \us
    tutorial built in the interpreter are good introductions to \us.

    This part does not aim at giving advanced programming
    techniques. Its only goal is to define the language and its
    libraries.
    %
  }
\item[sec:tools]
  Presentation and usage of the different tools available with the
  \urbi framework related to \us, such as the \urbi server, the
  command line client, \umake, \ldots

\item[sec:lang]
  Core constructs of the language and their behavior.

\item[sec:stdlib]
  Listing of all classes and methods provided in the standard library.

\item[sec:specs:ros] \urbi provides a set of tools to communicate with ROS
  (Robot Operating System). For more information about ROS, see
  \url{http://www.ros.org}.  \urbi, acting as a ROS node, is able to
  interact with the ROS world.

\item[sec:naming]
  Also known as ``The \urbi Naming Standard'': naming conventions in
  for standard hardware/software devices and components implemented as
  UObject and the corresponding slots/events to access them.

%\item[sec:sdk]
%  The \urbi software development kit that enable to
%  interact with \urbi from \Cxx.
\end{partDescription}


%%% Local Variables:
%%% mode: latex
%%% TeX-master: "../urbi-sdk"
%%% ispell-dictionary: "american"
%%% ispell-personal-dictionary: "../urbi.dict"
%%% fill-column: 76
%%% End:

%% Copyright (C) 2010, Gostai S.A.S.
%%
%% This software is provided "as is" without warranty of any kind,
%% either expressed or implied, including but not limited to the
%% implied warranties of fitness for a particular purpose.
%%
%% See the LICENSE file for more information.

\begin{partDescription}{part:specs}
  {
    %
    This part defines the specifications of the \us language version
    2.0. It defines the expected behavior from the \us interpreter,
    the standard library, and the \sdk. It can be used to check
    whether some code is valid, or browse \us or \Cxx \api for a
    desired feature. Random reading can also provide you with advanced
    knowledge or subtleties about some \us aspects.

    This part is not an \us tutorial; it is not structured in a
    progressive manner and is too detailed.  Think of it as a
    dictionary: one does not learn a foreign language by reading a
    dictionary. The \us Tutorial (\autoref{part:tut}), or the live \us
    tutorial built in the interpreter are good introductions to \us.

    This part does not aim at giving advanced programming
    techniques. Its only goal is to define the language and its
    libraries.
    %
  }
\item[sec:tools]
  Presentation and usage of the different tools available with the
  \urbi framework related to \us, such as the \urbi server, the
  command line client, \umake, \ldots

\item[sec:lang]
  Core constructs of the language and their behavior.

\item[sec:stdlib]
  Listing of all classes and methods provided in the standard library.

\item[sec:specs:ros] \urbi provides a set of tools to communicate with ROS
  (Robot Operating System). For more information about ROS, see
  \url{http://www.ros.org}.  \urbi, acting as a ROS node, is able to
  interact with the ROS world.

\item[sec:naming]
  Also known as ``The \urbi Naming Standard'': naming conventions in
  for standard hardware/software devices and components implemented as
  UObject and the corresponding slots/events to access them.

%\item[sec:sdk]
%  The \urbi software development kit that enable to
%  interact with \urbi from \Cxx.
\end{partDescription}


%%% Local Variables:
%%% mode: latex
%%% TeX-master: "../urbi-sdk"
%%% ispell-dictionary: "american"
%%% ispell-personal-dictionary: "../urbi.dict"
%%% fill-column: 76
%%% End:

\ifthen{\boolean{platforms}}
{
  %% Copyright (C) 2010, Gostai S.A.S.
%%
%% This software is provided "as is" without warranty of any kind,
%% either expressed or implied, including but not limited to the
%% implied warranties of fitness for a particular purpose.
%%
%% See the LICENSE file for more information.

\begin{partDescription}{part:specs}
  {
    %
    This part defines the specifications of the \us language version
    2.0. It defines the expected behavior from the \us interpreter,
    the standard library, and the \sdk. It can be used to check
    whether some code is valid, or browse \us or \Cxx \api for a
    desired feature. Random reading can also provide you with advanced
    knowledge or subtleties about some \us aspects.

    This part is not an \us tutorial; it is not structured in a
    progressive manner and is too detailed.  Think of it as a
    dictionary: one does not learn a foreign language by reading a
    dictionary. The \us Tutorial (\autoref{part:tut}), or the live \us
    tutorial built in the interpreter are good introductions to \us.

    This part does not aim at giving advanced programming
    techniques. Its only goal is to define the language and its
    libraries.
    %
  }
\item[sec:tools]
  Presentation and usage of the different tools available with the
  \urbi framework related to \us, such as the \urbi server, the
  command line client, \umake, \ldots

\item[sec:lang]
  Core constructs of the language and their behavior.

\item[sec:stdlib]
  Listing of all classes and methods provided in the standard library.

\item[sec:specs:ros] \urbi provides a set of tools to communicate with ROS
  (Robot Operating System). For more information about ROS, see
  \url{http://www.ros.org}.  \urbi, acting as a ROS node, is able to
  interact with the ROS world.

\item[sec:naming]
  Also known as ``The \urbi Naming Standard'': naming conventions in
  for standard hardware/software devices and components implemented as
  UObject and the corresponding slots/events to access them.

%\item[sec:sdk]
%  The \urbi software development kit that enable to
%  interact with \urbi from \Cxx.
\end{partDescription}


%%% Local Variables:
%%% mode: latex
%%% TeX-master: "../urbi-sdk"
%%% ispell-dictionary: "american"
%%% ispell-personal-dictionary: "../urbi.dict"
%%% fill-column: 76
%%% End:

}
%% Copyright (C) 2010, Gostai S.A.S.
%%
%% This software is provided "as is" without warranty of any kind,
%% either expressed or implied, including but not limited to the
%% implied warranties of fitness for a particular purpose.
%%
%% See the LICENSE file for more information.

\begin{partDescription}{part:specs}
  {
    %
    This part defines the specifications of the \us language version
    2.0. It defines the expected behavior from the \us interpreter,
    the standard library, and the \sdk. It can be used to check
    whether some code is valid, or browse \us or \Cxx \api for a
    desired feature. Random reading can also provide you with advanced
    knowledge or subtleties about some \us aspects.

    This part is not an \us tutorial; it is not structured in a
    progressive manner and is too detailed.  Think of it as a
    dictionary: one does not learn a foreign language by reading a
    dictionary. The \us Tutorial (\autoref{part:tut}), or the live \us
    tutorial built in the interpreter are good introductions to \us.

    This part does not aim at giving advanced programming
    techniques. Its only goal is to define the language and its
    libraries.
    %
  }
\item[sec:tools]
  Presentation and usage of the different tools available with the
  \urbi framework related to \us, such as the \urbi server, the
  command line client, \umake, \ldots

\item[sec:lang]
  Core constructs of the language and their behavior.

\item[sec:stdlib]
  Listing of all classes and methods provided in the standard library.

\item[sec:specs:ros] \urbi provides a set of tools to communicate with ROS
  (Robot Operating System). For more information about ROS, see
  \url{http://www.ros.org}.  \urbi, acting as a ROS node, is able to
  interact with the ROS world.

\item[sec:naming]
  Also known as ``The \urbi Naming Standard'': naming conventions in
  for standard hardware/software devices and components implemented as
  UObject and the corresponding slots/events to access them.

%\item[sec:sdk]
%  The \urbi software development kit that enable to
%  interact with \urbi from \Cxx.
\end{partDescription}


%%% Local Variables:
%%% mode: latex
%%% TeX-master: "../urbi-sdk"
%%% ispell-dictionary: "american"
%%% ispell-personal-dictionary: "../urbi.dict"
%%% fill-column: 76
%%% End:

\end{description}

%% Redefine this environment so that next time the */abstract.tex
%% files are read, they create the part instead of referencing to it.

\renewenvironment{partDescription}[2]
{%
  \chapter*{About This Part}
  #2
  \begin{description}%
    \let\itemOrig\item%
    \renewcommand{\item}[1][]{\itemOrig[~~\autoref{##1} --- \nameref{##1}]~\\}%
  }{%
  \end{description}%
}


%%% Local Variables:
%%% mode: latex
%%% TeX-master: "urbi-sdk"
%%% ispell-dictionary: "american"
%%% ispell-personal-dictionary: "urbi.dict"
%%% fill-column: 76
%%% End:
