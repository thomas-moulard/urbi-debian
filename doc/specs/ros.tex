%% Copyright (C) 2010, Gostai S.A.S.
%%
%% This software is provided "as is" without warranty of any kind,
%% either expressed or implied, including but not limited to the
%% implied warranties of fitness for a particular purpose.
%%
%% See the LICENSE file for more information.

\chapter{Communication with ROS}
\setHtmlFileName{ros}
\label{sec:specs:ros}

This chapter is not an introduction to using ROS from \urbi, see
\autoref{sec:tut:ros} for a tutorial.

\urbi provides a set of tools to communicate with ROS (Robot Operating
System). For more information about ROS, please refer to
\url{http://www.ros.org}.  \urbi, acting as a ROS node, is able to interact
with the ROS world.


\paragraph{Requirements}

You need to have installed ROS (possibly a recent version), and compiled all
of the common ROS tools (\command{rxconsole}, \command{roscore},
\command{roscpp}, \ldots).

You also need to have a few environment variables set, normally provided
with ROS installation: \env{ROS\_ROOT}, \env{ROS\_MASTER\_URI} and
\env{ROS\_PACKAGE\_PATH}.


\paragraph{Usage}

The classes are implemented as uobjects (see \autoref{part:uobject}):
\refObject{Ros.Service}, \refObject{Ros.Topic}, and \refObject{Ros}.

This module is loaded automatically if \env{ROS\_ROOT} is set  in
your environment. If \command{roscore} is not launched, you will be noticed
by a warning and \urbi will check regularly for \command{roscore}.

If for any reason you need to load this module manually, use:
\begin{urbiunchecked}
loadModule("urbi/ros");
\end{urbiunchecked}


\section{Ros}
\defObject{Ros}

This object provides some handy tools to know the status of \command{roscore},
to list the different nodes, topics, services, \ldots.  It also serves as a
namespace entry point for ROS entities, such as \refObject{Ros.Topic} and so
forth.

\subsection{Construction}

There is no construction, since this class only provides a set of tools
related to ROS in general, or the current node (which is unique per instance
of \urbi).

\subsection{Slots}

\begin{urbiscriptapi}
\item[checkMaster]
  Whether \command{roscore} is accessible.

\item[name] The name of the current node associated with \urbi (as a
  string).  The name of an \urbi node is generally composed of
  \file{/urbi\_+random} sequence.

\item[nodes]%
  A dictionary containing all the nodes known by \command{roscore}. Each key
  of the dictionary is a node name. The associated value is an other
  dictionary, with the following keys: \lstinline{publish},
  \lstinline{subscribe}, \lstinline{services}.  Each of these keys is
  associated with a list of topics or services.
\begin{urbiassert}
"/rosout" in Ros.nodes;
Ros.name in Ros.nodes;
\end{urbiassert}


\item[Service]%
  The \refObject{Ros.Service} class.

\item[services]%
  A dictionary containing all the services known by \command{roscore}. Each
  key is the name of a service, and the associated value is a list of nodes
  that provide this service.
\begin{urbiscript}
//#roscore
var services = Ros.services|
var name = Ros.name|;
assert
{
         "/rosout/get_loggers" in services;
    "/rosout/set_logger_level" in services;
       (name + "/get_loggers") in services;
  (name + "/set_logger_level") in services;
};
\end{urbiscript}

\item[Topic]%
  The \refObject{Ros.Topic} class.

\item[topics]%
  A dictionary containing all the topics advertised to
  \command{roscore}. Each key is the name of a topic, and the associated
  value is an other dictionary, with the following keys:
  \lstinline{subscribers}, \lstinline{publishers}, \lstinline{type}.
\begin{urbiscript}
var topics = Ros.topics|;
topics.keys;
[03316144] ["/rosout_agg"]
topics["/rosout_agg"];
[03325634] ["publishers" => ["/rosout"], "subscribers" => [], "type" => "roslib/Log"]
\end{urbiscript}
\end{urbiscriptapi}


\section{Ros.Topic}
\defObject{Ros.Topic}

This \UObject provides a handy way to communicate through ROS topics, by
subscribing to existent topics or advertising to them.

\subsection{Construction}

To create a new topic, call \lstinline{Ros.topic.new} with a string (the
name of the topic you want to subscribe to / advertise on).

The topic name can only contain alphanumerical characters, \samp{/} and
\samp{\_}, and cannot be empty.  If the topic name is invalid, an exception
is thrown and the topic is not created.

Until you decide what you want to do with your topic (\refSlot{subscribe} or
\refSlot{advertise}), you are free to call \lstinline{init} to change its
name.


\subsection{Slots}

Some slots on this \UObject have no interest once the type of instance is
determined. For example, you cannot call unsubscribe if you advertise, and
in the same way you cannot call publish if you subscribed to a topic.

\subsubsection{Common}

\begin{urbiscriptapi}

\item[name]%
  The name of the topic you provided in \lstinline{init}.
\begin{urbiscript}
Ros.Topic.new("/test").name;
[00011644] "/test"
\end{urbiscript}


\item[subscriberCount]%
  The number of subscribers for the topic given in \lstinline{init}; 0 if
  the topic is not registered.

\item[structure]%
  Once \refSlot{advertise} or \refSlot{subscribe} has been called, this slot
  contains a template of the message type, with default values for each type
  (empty strings, zeros, \ldots). This template is made of
  \refObject[s]{Dictionary}.
\begin{urbiscript}
var logTopic = Ros.Topic.new("/rosout_agg")|
logTopic.subscribe|;
logTopic.structure.keys;
[00133519] ["file", "function", "header", "level", "line", "msg", "name", "topics"]
\end{urbiscript}
\end{urbiscriptapi}


\subsubsection{Subscription}

\begin{urbiscriptapi}

\item[subscribe]%
  Subscribe to the provided topic. Throw an exception it doesn't exist, or
  if the type advertised by ROS for this topic does not exist.  From the
  call of this method, a direct connection is made between the advertiser
  and the subscriber which starts deserializing the received messages.

\item[unsubscribe]%
  Unsubscribe from a previously subscribed channel, and set the state of the
  instance as if \lstinline{init} was called but not subscribe.

\item[onMessage]%
  Event triggered when a new message is received from a subscribed channel,
  the payload of this event contains the message.

\begin{urbiunchecked}
var t = Ros.Topic.new("/test")|;
msg: at (t.onMessage?(var m)) echo(m);
t.subscribe;
\end{urbiunchecked}
\end{urbiscriptapi}


\subsubsection{Advertising}

\begin{urbiscriptapi}

\item[advertise](<type>)%
  Tells ROS that this node will advertise on the topic given in
  \lstinline{init}, with the type \var{type}. \var{type} must be a valid ROS
  type, such as \samp{roslib/Log}. If \var{type} is an empty string, the
  method will try to check whether a node is already advertising on this
  topic, and its type.
\begin{urbiscript}
var stringPub = Ros.Topic.new("/mytest")|;
stringPub.advertise("std_msgs/String");
stringPub.structure;
[00670809] ["data" => ""]
\end{urbiscript}


\item[onConnect](<name>)%
  This event is triggered when the current instance is advertising on a
  topic and a node subscribes to this topic. The payload \var{name} is the
  name of the node that subscribed.
\begin{urbiunchecked}
var p = Ros.Topic.new("/test/publisher")|;
con: at (p.onConnect?(var n))
  echo(n + " has subscribed to " + p.name);
p.advertise("roslib/Log");
\end{urbiunchecked}

\item[onDisconnect](<name>)%
  This event is triggered when the current instance is advertising on a
  topic, and a node unsubscribes from this topic. The payload is the name of
  the node that unsubscribed.

\item[publish](<message>)%
  Publish \var{message} to the current topic. This method is usable only if
  \refSlot{advertise} was called. The message must follow the same structure
  as the one in the slot \lstinline{structure}, or it will throw an
  exception telling which key is missing / wrong in the dictionary.

\item['<<'](<message>)%
  An alias for \refSlot{publish}.
\begin{urbiunchecked}
stringPub << ["data" => "Hello world!"];
\end{urbiunchecked}


\item[unadvertise]%
  Tells ROS that we stop publishing on this topic. The \UObject then gets back
  to a state where it could call \lstinline{init}, \refSlot{subscribe} or
  \refSlot{advertise}.
\end{urbiscriptapi}


\subsection{Example}

This is a typical example of the creation of a publisher, a subscriber, and
message transmission between both of them.

First we need to declare our Publisher, the topic name is \file{/example},
and the type of message that will be sent on this topic is
\samp{std\_msgs/String}. This type contains a single field called
\samp{data}, holding a string.  We also set up handlers for
\refSlot{onConnect} and \refSlot{onDisconnect} to be noticed when someone
subscribes to us.

\begin{urbiscript}
var publisher = Ros.Topic.new("/example")|
con: at (publisher.onConnect?(var name))
  echo(name[0,5] + " is now listening on " + publisher.name);
dcon: at (publisher.onDisconnect?(var name))
  echo(name[0,5] + " is no longer listening on " + publisher.name);
publisher.advertise("std_msgs/String");
\end{urbiscript}

Then we subscribe to the freshly created topic, and for each message, we
display the \samp{data} section (which is the content of the
message). Thanks to the previous \lstinline{at} above, a message is
displayed at subscription time.

\begin{urbiscript}
var subscriber = Ros.Topic.new("/example")|
msg: at (subscriber.onMessage?(var m))
  echo(m["data"]);
subscriber.subscribe;
[00026580] *** /urbi is now listening on /example

// The types should be the same, ensure that.
assert { subscriber.structure == publisher.structure };
\end{urbiscript}

\begin{comment}
\begin{urbiscript}
// A small delay to let the "is now listening" message arrives
// before the next interaction.
sleep(500ms);
\end{urbiscript}
\end{comment}

Now we can send a message, and get it back through the at in the section
above. To do this we first copy the template structure and then fill the
field \samp{data} with our message.

\begin{urbiscript}
var message = publisher.structure.new;
[00098963] ["data" => ""]
message["data"] = "Hello world!"|;

// publish the message.
publisher << message;
// Leave some time to asynchronous communications before shutting down.
sleep(200ms);
[00098964] *** Hello world!

subscriber.unsubscribe;
[00252566] *** /urbi is no longer listening on /example
\end{urbiscript}



\section{Ros.Service}
\defObject{Ros.Service}

This \UObject provides a handy way to call services provided by other ROS
nodes.


\subsection{Construction}

To create a new instance of this object, call \lstinline{Ros.Service.new}
with a string representing which service you want to use, and a Boolean
stating whether the connection between you and the service provider should
be kept opened (pass \lstinline{true} for better performances on multiple
requests).

The service name can only contain alphanumerical characters, \samp{/},
\samp{\_}, and cannot be empty. If the service name is invalid, an exception
is thrown, and the object is not created.

Then if the service does not exist, an other exception is thrown. Since the
initialization is asynchronous internally, you need to wait for
\lstinline|\var{service}.initialized| to be true to be able to call
\lstinline{request}.


\subsection{Slots}

\begin{urbiscriptapi}

\item[initialized]%
  Boolean.  Whether the method \var{request} is ready to be called, and
  whether \refSlot{resStruct} and \refSlot{reqStruct} are filled.
\begin{urbiscript}
var logService = Ros.Service.new("/rosout/get_loggers", false)|;
waituntil(logService.initialized);
\end{urbiscript}

\item[name]%
  The name of the current service.
\begin{urbiscript}
logService.name;
[00036689] "/rosout/get_loggers"
\end{urbiscript}

\item[resStruct]%
  Get the template of the response message type, with default values for
  each type (empty strings, zeros, \ldots). This template is made of
  dictionaries.
\begin{urbiscript}
logService.resStruct;
[00029399] ["loggers" => []]
\end{urbiscript}

\item[reqStruct]%
  Get the template of the request message type, with default values for
  each type (empty strings, zeros, \ldots). This template is made of
  dictionaries.
\begin{urbiscript}
logService.reqStruct;
[00029399] [ => ]
\end{urbiscript}


\item[request](<req>)%
  Synchronous call to the service, providing \var{req} as your request
  (following the structure of \refSlot{reqStruct}). The returned value is a
  dictionary following the structure of \refSlot{resStruct}, containing the
  response to your request.
\begin{urbiscript}
for (var item in logService.request([=>])["loggers"])
  echo(item);
[00236349] *** ["level" => "INFO", "name" => "ros"]
[00236349] *** ["level" => "INFO", "name" => "ros.roscpp"]
[00236350] *** ["level" => "WARN", "name" => "ros.roscpp.superdebug"]
[00236350] *** ["level" => "DEBUG", "name" => "roscpp_internal"]
\end{urbiscript}

\end{urbiscriptapi}

%%% Local Variables:
%%% mode: latex
%%% TeX-master: "../urbi-sdk"
%%% ispell-dictionary: "american"
%%% ispell-personal-dictionary: "../urbi.dict"
%%% fill-column: 76
%%% End:
